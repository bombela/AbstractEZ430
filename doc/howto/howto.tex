% doc/howto/howto.tex
% Copyright © 2010 François-Xavier 'Bombela' Bourlet <bombela@gmail.com>

\documentclass[11pt,a4paper]{article}

\usepackage[T1]{fontenc}
\usepackage[utf8]{inputenc}
\usepackage[english]{varioref}
\usepackage[pdftex]{color,graphicx}
\usepackage{listings}
\usepackage{lmodern}
\usepackage{courier}
\usepackage{caption}
\usepackage[english]{babel}

\lstset{
	basicstyle=\footnotesize\ttfamily,
	tabsize=2,
	extendedchars=true,
	breaklines=true,
	keywordstyle=\textbf,
	frame=b,
	stringstyle=\ttfamily,
	showspaces=false,
	showtabs=false,
	xleftmargin=17pt,
	framexleftmargin=17pt,
	framexrightmargin=5pt,
	framexbottommargin=4pt,
	showstringspaces=false
}

\DeclareCaptionFont{white}{\color{white}}

\DeclareCaptionFormat{listing}{
	\colorbox[rgb]{0.6, 0.6, 0.6}{\parbox{\textwidth}{\hspace{15pt}#1#2#3}}
	}

\captionsetup[lstlisting]{
	format=listing,
	labelfont=white,
	textfont=white,
	singlelinecheck=false,
	margin=0pt,
	font={bf,footnotesize}
	}

\title{HOWTO using AbstractEZ430 library with C++ and Python language}
\author{François-Xavier 'Bombela' Bourlet <bombela@gmail.com>}

\lstnewenvironment{C++}{
	\lstset{
		language=C++,
		title=C++
	}
}{}

\lstnewenvironment{python}{
	\lstset{
		language=python,
		title=Python
	}
}{}

\makeindex
\begin{document}

\maketitle
\newpage

\tableofcontents
\newpage

\section{Introduction}

This HOWTO will present to you the best and simple way
to use the AbstractEZ430 library.

You will learn to initiate a communication to an access point,
and next how to communicate with the watch.

The two languages actually supported by the library
are C++ (natively supported) and Python (binding).

The API (Application Programming Interface) is OO (Oriented Object)
both supported by C++ and Python.

The API propose tree classes:
\begin{itemize}
	\item \textbf{AccessPoint},
			Provide the necessary for controlling the RF (Radio Frequency) access point.
			Need a serial port to operate on it.
	\item \textbf{Service},
			The most basic way to communicate with the watch proposed
			by the API. This is not indented to be user friendly.
			AccessPoint class provide a specialized version of this service
			to be used by the Watch class.
	\item \textbf{Watch},
			an high-level class to communication with the watch via the RF access point.
			This class need a Service implementation to deal with the watch,
			like the Service gibed by the AccessPoint class.
			The Watch class is only a wrapper class on top of the Service class,
			for providing an easier interface to the library user.
\end{itemize}

\break

\section{Using RF access point}

First you have to import all the necessary library definition.

\begin{C++}
#include <iostream>
#include <ez430_watch/ap.h>

using namespace ez430;

int main()
{
	AccessPoint ap;

	try {
		ap.open("/dev/ttyACMO");
	}
	catch (const std::exception& e) {
		std::cerr << "Error: " << e.what() << std::endl;
		return -1;
	}

	if (!ap.isOpen()) {
		std::cerr << "Access point must be open !"
		return -2;
	}
	return 0;
}
\end{C++}

\begin{python}
#!/usr/bin/env python

import ez430

ap = ez430.AccessPoint()
ap.open("/dev/ttyACM0")
if not ap.isOpen():
	print "Access point must be open !"
	exit(-2)
\end{python}

\break
Now, if you want use any service from the access point,
we have to start-it, startRadio() is your friend:

\begin{C++}
std::cout << "Starting radio..." << std::endl;
ap.startRadio();
while (ap.getRadioState() == AccessPoint::STOPPED) {
	std::cout << "Waiting for radio to be up..." << std::endl;
	// This loop is really nasty, this is just a (bad) example.
}
std::cout << "Radio started!" << std::endl;
\end{C++}

\begin{python}
ap.startRadio()
while ap.getRadioState() == ez430.STOPPED:
	print "Waiting for radio tu be up..."
	# This loop is really nasty, this is just a (bad) example.
print "Radio started!"
\end{python}

\break
\section{Using RF access point port detection}

You can use the automatic access point detection port
for removing the necessity to provide it manually
(or asking the user of your application).

\textit{Note: actually the detection is only performed on GNU/Linux
platform, on other system the detection will be empty.}

The function probePorts() retrieve all the access points discovered
on the system:

\begin{C++}
#include <iostream>
#include <ez430_watch/ap.h>

using namespace ez430;

int main()
{
	const std::vector<std::string>& ports = AccessPoint::probePorts();

	if (ports.size() == 0) {
		std::cout << "No ports discovered :'('" << std::endl;
		return -1;
	}

	AccessPoint ap;
	for (std::vector<std::string>::const_iterator
		i = ports.begin(); i != ports.end(); ++i)
	{
		std::cout << "opening: " << *i << " ..." << std::endl;
		try { ap.open(*i); } catch (...) {}
		if (!ap.isOpen()) {
			std::cout << "unable to open " << *i << std::endl;
			continue;
		}
		std::cout << std::hex << "product id: "
				<< ap.getProductId() << std::dec << std::endl;
		std::cout << "current radio state: " << ap.getRadioState() << std::endl;
	}
	return 0;
}
\end{C++}

\begin{python}
#!/usr/bin/env python

import ez430

ports = ez430.probePorts()

if len(ports) == 0:
	print "No ports discovered :'('"
	exit(-1)

ap = ez430.AccessPoint()
foreach i in ports:
	print "opening: %s" % i
	try:
		ap.open(i);
	except:
		pass
	if not ap.isOpen():
		print "unable to open "
		continue
	print "product id: %x" % ap.getProductId()
	print "current radio state: %s" % ap.getRadioState()
\end{python}

\section{Using Watch}

%\section{Listings}
%\lstlistoflistings

\end{document}
